\input{input_accessability.tex}
\documentclass[11pt,
  english,
  a4paper,
]{article}
\usepackage{sa4ss}
\usepackage{amsmath,amssymb,array}
\usepackage{booktabs}

% From tagged-template.latex
\usepackage{lmodern}
\usepackage{ifxetex,ifluatex}
\ifnum 0\ifxetex 1\fi\ifluatex 1\fi=0 % if pdftex
  \usepackage[T1]{fontenc}
  \usepackage[utf8]{inputenc}
  \usepackage{textcomp} % provide euro and other symbols
\else % if luatex or xetex
  \usepackage{unicode-math}
  \defaultfontfeatures{Scale=MatchLowercase}
  \defaultfontfeatures[\rmfamily]{Ligatures=TeX,Scale=1}
\fi

% Use upquote if available, for straight quotes in verbatim environments
\IfFileExists{upquote.sty}{\usepackage{upquote}}{}
\IfFileExists{microtype.sty}{% use microtype if available
  \usepackage[]{microtype}
  \UseMicrotypeSet[protrusion]{basicmath} % disable protrusion for tt fonts
}{}
\makeatletter
\@ifundefined{KOMAClassName}{% if non-KOMA class
  \IfFileExists{parskip.sty}{%
    \usepackage{parskip}
  }{% else
    \setlength{\parindent}{0pt}
    \setlength{\parskip}{6pt plus 2pt minus 1pt}}
}{% if KOMA class
  \KOMAoptions{parskip=half}}
\makeatother
\usepackage{xcolor}
\IfFileExists{xurl.sty}{\usepackage{xurl}}{} % add URL line breaks if available
\hypersetup{
  pdflang={en},
  hidelinks,
  pdfcreator={LaTeX via pandoc}}
\urlstyle{same} % disable monospaced font for URLs
\usepackage{longtable}
% Correct order of tables after \paragraph or \subparagraph
\usepackage{etoolbox}
\makeatletter
\patchcmd\longtable{\par}{\if@noskipsec\mbox{}\fi\par}{}{}
\makeatother
% Allow footnotes in longtable head/foot
\IfFileExists{footnotehyper.sty}{\usepackage{footnotehyper}}{\usepackage{footnote}}
\makesavenoteenv{longtable}
\usepackage{graphicx}
\makeatletter
\def\maxwidth{\ifdim\Gin@nat@width>\linewidth\linewidth\else\Gin@nat@width\fi}
\def\maxheight{\ifdim\Gin@nat@height>\textheight\textheight\else\Gin@nat@height\fi}
\makeatother
% Scale images if necessary, so that they will not overflow the page
% margins by default, and it is still possible to overwrite the defaults
% using explicit options in \includegraphics[width, height, ...]{}
\setkeys{Gin}{width=\maxwidth,height=\maxheight,keepaspectratio}
% Set default figure placement to htbp
\makeatletter
\def\fps@figure{htbp}
\makeatother
\setlength{\emergencystretch}{3em} % prevent overfull lines
\providecommand{\tightlist}{%
  \setlength{\itemsep}{0pt}\setlength{\parskip}{0pt}}
\setcounter{secnumdepth}{5}
\ifxetex
  % Load polyglossia as late as possible: uses bidi with RTL langages (e.g. Hebrew, Arabic)
  \usepackage{polyglossia}
  \setmainlanguage[]{english}
\else
  \usepackage[shorthands=off,main=english]{babel}
\fi

\providecommand{\tightlist}{%
  \setlength{\itemsep}{0pt}\setlength{\parskip}{0pt}}


\date{}
\newcommand{\trTitle}{}
\newcommand{\trYear}{2021}
\newcommand{\trMonth}{June}
\newcommand{\trAuthsLong}{}
\newcommand{\trAuthsBack}{}
\newcommand{\trCitation}{
\begin{hangparas}{1em}{1}
\trAuthsBack{}. \trYear{}. \trTitle{}. Pacific Fisheries Management Council, Portland, Oregon. \pageref{LastPage}{}\,p.
\end{hangparas}}

\AtBeginDocument{\tagstructbegin{tag=Document}}
\AtEndDocument{\tagstructend}
\pretocmd{\maketitle}{\tagstructbegin{tag=H1}\tagmcbegin{tag=H1}}{}{}
\apptocmd{\maketitle}{\tagmcend\tagstructend}{}{}

\begin{document}

%%%%% Frontmatter %%%%%

% Footnote symbols in front matter
\renewcommand*{\thefootnote}{\fnsymbol{footnote}}

\small
\thispagestyle{empty}
\pagenumbering{roman}
\noindent
\begin{center}
\title{}
% \textnormal{\MakeTextUppercase{\trTitle{}}}
\vspace{1.5cm}
{\Large\textbf\newline{}}
\vfill
by\\
\vfill
\vfill
\trMonth{} \trYear{}
\end{center}
\clearpage

% Fourth page: Colophon
\thispagestyle{empty}
\vspace*{\fill}
\begin{center}
\copyright{} Pacific Fisheries Management Council, \trYear{}\\
\end{center}
\par
\bigskip
\noindent
Correct citation for this publication:
\bigskip
\par
\trCitation{}
\clearpage

% Add TOC to pdf bookmarks (clickable pdf)
\pdfbookmark[1]{\contentsname}{toc}

% Table of contents page, lists of figures and tables
\tableofcontents\clearpage
%\listoffigures \listoftables \clearpage
\label{TRlastRoman}
\clearpage

% Table of contents
\newpage
\thispagestyle{empty} % to remove page number

% Settings for the main document
\pagenumbering{arabic}  % Regular page numbers
\pagestyle{plain}  % No page number on first page of main document, use 'empty'
\renewcommand*{\thefootnote}{\arabic{footnote}}  % Back to numeric footnotes
\setcounter{footnote}{0}  % And start at 1
\renewcommand{\headrulewidth}{0.5pt}
\renewcommand{\footrulewidth}{0.5pt}
%\pagestyle{fancy}\fancyhead[c]{Draft: Do not cite or circulate}

\newcommand{\lt}{\ensuremath <}
\newcommand{\gt}{\ensuremath >}

%Define cslreferences environment, required by pandoc 2.8
%https://github.com/rstudio/rmarkdown/issues/1649
\newlength{\cslhangindent}
\setlength{\cslhangindent}{1.5em}
\newenvironment{cslreferences}%
  {\setlength{\parindent}{0pt}%
  \everypar{\setlength{\hangindent}{\cslhangindent}}\ignorespaces}%
  {\par}

\pagenumbering{roman}

\renewcommand{\thetable}{\roman{table}}
\renewcommand{\thefigure}{\roman{figure}}

\setlength\parskip{0.5em plus 0.1em minus 0.2em}

\tagstructbegin{tag=H1}\tagmcbegin{tag=H1}

\hypertarget{executive-summary}{%
\section*{Executive Summary}\label{executive-summary}}
\addcontentsline{toc}{section}{Executive Summary}

\leavevmode\tagmcend\tagstructend

\tagstructbegin{tag=H2}\tagmcbegin{tag=H2}

\hypertarget{stock}{%
\subsection*{Stock}\label{stock}}
\addcontentsline{toc}{subsection}{Stock}

\leavevmode\tagmcend\tagstructend

\tagstructbegin{tag=P}\tagmcbegin{tag=P}

This assessment reports the status of Dover sole (\emph{Microstomus pacificus}) off the U.S. west coast using data through 2020. Dover sole are also harvested from the waters off the Canadian coast and in the Gulf of Alaska, and although those catches were not included in this assessment, it is not certain if those populations contribute to the biomass of Dover sole off the U.S. west coast. Dover sole exhibit complex seasonal and ontogenetic movement, moving to deeper waters based on size but also shifting seasonally, moving from shallower feeding grounds on the continental shelf during the summer months to deeper spawning habitat on the outer continental shelf and slope in the winter. However, the specific mechanisms that drive stock structure and related variability over space and time, are not well understood.

\leavevmode\tagmcend\tagstructend\par

\tagstructbegin{tag=H2}\tagmcbegin{tag=H2}

\hypertarget{landings}{%
\subsection*{Landings}\label{landings}}
\addcontentsline{toc}{subsection}{Landings}

\leavevmode\tagmcend\tagstructend

\tagstructbegin{tag=P}\tagmcbegin{tag=P}

Dover sole were first landed in California in the early part of the 20th century with landings beginning in Oregon and Washington in the 1940's (Figure \ref{fig:es-catch}). Landings remained relatively constant throughout the 1950s and 1960s before increasing rapidly into the early 1990s. Subsequently, the landings declined by nearly 60 percent in California and Oregon/Washington until 2007 when harvest guidelines increased the allowable catch leading to increased landings between 2007 - 2010. Since 2011, landings have been steadily decreasing, where the landings in 2020 is the lowest on record since the 1940s (Table \ref{tab:removalsES}). There are multiple factors that have led to the recent low landings of Dover sole (e.g., co-occurrence with constraining stocks, market forces).

\leavevmode\tagmcend\tagstructend\par

\tagstructbegin{tag=P}\tagmcbegin{tag=P}

Groundfish trawl fisheries account for the majority of Dover sole landings off the West Coast, with fixed gears, shrimp trawls, and recreational fisheries collectively make up a very small amount of fishing mortality (less that 1 percent of the total). Some discarding of Dover sole has occurred in these fisheries, primarily prior to the implementation of the Individual Fishing Quota (IFQ) Catch Shares Program in 2011. Discard mortality was estimated within the model based on data of discarding rates and lengths across time. Landings and the estimates of total mortality are reported (Table \ref{tab:removalsES}).

\leavevmode\tagmcend\tagstructend\par

\clearpage

\begingroup\fontsize{10}{12}\selectfont
\begingroup\fontsize{10}{12}\selectfont

\begin{longtable}[t]{r>{\centering\arraybackslash}p{2.2cm}>{\centering\arraybackslash}p{2.2cm}>{\centering\arraybackslash}p{2.2cm}>{\centering\arraybackslash}p{2.2cm}}
\caption{\label{tab:removalsES}Recent landings by fleet, total landings summed across fleets, and the total mortality including discards.}\\
\toprule
Year & California & Oregon/Washington & Total Catch & Total Mortality\\
\midrule
\endfirsthead
\caption[]{Recent landings by fleet, total landings summed across fleets, and the total mortality including discards. \textit{(continued)}}\\
\toprule
Year & California & Oregon/Washington & Total Catch & Total Mortality\\
\midrule
\endhead

\endfoot
\bottomrule
\endlastfoot
2009 & 3163.46 & 8583.79 & 11747.25 & 12451.17\\
2010 & 2620.38 & 7771.50 & 10391.88 & 10999.29\\
2011 & 2401.08 & 5381.29 & 7782.37 & 7893.18\\
2012 & 2160.60 & 5167.29 & 7327.89 & 7429.72\\
2013 & 2217.77 & 5752.41 & 7970.18 & 8077.92\\
2014 & 1954.98 & 4494.25 & 6449.23 & 6543.10\\
2015 & 1892.58 & 4434.15 & 6326.73 & 6354.50\\
2016 & 1808.26 & 5510.11 & 7318.37 & 7349.81\\
2017 & 2196.85 & 5694.75 & 7891.60 & 7925.06\\
2018 & 1640.28 & 4780.99 & 6421.27 & 6447.41\\
2019 & 1397.44 & 4369.42 & 5766.86 & 5789.61\\
2020 & 1616.99 & 3070.65 & 4687.64 & 4706.57\\*
\end{longtable}
\endgroup{}
\endgroup{}


\tagstructbegin{tag=Figure,alttext={Landings by fleet used in the base model where catches in metric tons by fleet are stacked.}}\tagmcbegin{tag=Figure}

\begin{figure}
\centering
\includegraphics[width=1\textwidth,height=1\textheight]{C:/Assessments/2021/dover_sole_2021/models/7.0.1_base/plots/catch2 landings stacked.png}
\caption{Landings by fleet used in the base model where catches in metric tons by fleet are stacked.\label{fig:es-catch}}
\end{figure}

\tagmcend\tagstructend

\clearpage

\tagstructbegin{tag=H2}\tagmcbegin{tag=H2}

\hypertarget{data-and-assessment}{%
\subsection*{Data and Assessment}\label{data-and-assessment}}
\addcontentsline{toc}{subsection}{Data and Assessment}

\leavevmode\tagmcend\tagstructend

\tagstructbegin{tag=P}\tagmcbegin{tag=P}

This stock assessment for Dover sole off the west coast of the U.S. was developed using the length- and age-structured model Stock Synthesis (version 3.30.16). The previous stock assessment of Dover sole was conducted in 2011 and estimated the stock to be increasing with a stock status determination of 84 percent of virgin (or unfished) spawning biomass at the beginning of 2011. During the development of this assessment, model specifications including fleet structure, landings, data, and model structural assumptions were re-evaluated. Similar to the previous assessment, a single coastwide population was modeled allowing for area-specific fleets and separate growth and mortality parameters for each sex (i.e., a two-sex model). The model time domain is 1911 to 2020, with a 12 year forecast beginning in 2021.

\leavevmode\tagmcend\tagstructend\par

\tagstructbegin{tag=P}\tagmcbegin{tag=P}

All the data sources included in the base model for Dover sole have been re-evaluated for this assessment, including improvements and updates in the data (and associated analyses) that were used in the previous assessment. Estimate of landings prior to the mid-1980s have also been updated using the new historical catch reconstruction time series for Oregon. Survey data from the \gls{afsc} and \gls{nwfsc} have been used to construct four sets of relative abundance indices, each spanning different time periods, were independently developed using a spatio-temporal delta-generalized linear mixed model (i.e., VAST).

\leavevmode\tagmcend\tagstructend\par

\tagstructbegin{tag=P}\tagmcbegin{tag=P}

The definition of fishing fleets changed in this assessment relative to those in the 2011 assessment. Two fishing fleets are now defined in the model: 1) a combined gear California fleet and 2) a combined gear Oregon/Washington fleet. The fleet grouping for Oregon and Washington was suggested by State representatives during the pre-assessment data meeting because of similarities in fishing across this region while also avoiding the inherent difficulties associated with separating data between Oregon and Washington due to the intermixing of fishing and landing locations across state boundaries.

\leavevmode\tagmcend\tagstructend\par

\tagstructbegin{tag=P}\tagmcbegin{tag=P}

This assessment integrates data and information from multiple sources into one modeling framework. Specifically, the assessment uses landings data and discard estimates; survey indices of abundance, length- and/or age-composition data for each fishery or survey (with conditional age-at-length data used for the \gls{s-nwslope} and \gls{s-wcgbt}); information on weight-at-length, maturity-at-length, and fecundity-at-length; information on natural mortality and the steepness of the Beverton-Holt stock-recruitment relationship; and estimates of ageing error. The base model was tuned to account for the weighting of composition data as well as the specification of recruitment variance and recruitment bias adjustments. Estimates of recruitment at equilibrium spawning biomass ({\tagstructbegin{tag=Formula}\tagmcbegin{tag=Formula}\(R_0\)\leavevmode\tagmcend\tagstructend}), annual recruitment deviations, sex-specific length-based selectivity of the fisheries and surveys, retention for each of the fishery fleets, catchability of the surveys, sex-specific growth, the time series of spawning biomass, age and size structure, and current and projected future stock status are derived outputs of the model.

\leavevmode\tagmcend\tagstructend\par

\tagstructbegin{tag=P}\tagmcbegin{tag=P}

Multiple sources of uncertainty are explicitly included in this assessment, including parameter uncertainty using prior distributions, observational uncertainty through standard deviations of survey estimates, and model uncertainty through a comprehensive sensitivity analyses to data source and model structural assumptions. A base model was selected that best fit the observed data while concomitantly balancing the desire to capture the central tendency across those sources of uncertainty, ensure model realism and tractability, and promote robustness to potential model misspecification.

\leavevmode\tagmcend\tagstructend\par

\tagstructbegin{tag=H2}\tagmcbegin{tag=H2}

\hypertarget{stock-biomass}{%
\subsection*{Stock Biomass}\label{stock-biomass}}
\addcontentsline{toc}{subsection}{Stock Biomass}

\leavevmode\tagmcend\tagstructend

\tagstructbegin{tag=P}\tagmcbegin{tag=P}

The terms ``spawning output'' and ``spawning biomass'' are used interchangeably in this document in reference to total female spawning biomass. For the purposes of this assessment, female spawning biomass is assumed to be proportional to egg and larval production (i.e., spawning output). The estimated spawning biomass at the beginning of 2021 was 232,065 mt (\textasciitilde95 percent asymptotic intervals: 154,153 to 309,977 mt, Table \ref{tab:ssbES} and Figure \ref{fig:es-ssb}), which when compared to unfished spawning biomass (294,070 mt) equates to a relative stock status level of 79 percent (\textasciitilde95 percent asymptotic intervals: 71 to 87 percent, Figure \ref{fig:es-depl}). Overall, spawning stock biomass has steadily declined from near unfished levels in the 1940s to a time series low of 60 percent of unfished levels in 1994 following high landings in the 1980s and early 1990s. Over the past two decades, spawning stock biomass has generally been increasing as total landings have decreased. The stock is estimated to be well above the management target of {\tagstructbegin{tag=Formula}\tagmcbegin{tag=Formula}\(SB_{25\%}\)\leavevmode\tagmcend\tagstructend} in 2021 and has remained well above the target throughout the time series (Table \ref{tab:ssbES} and Figure \ref{fig:es-depl}).

\leavevmode\tagmcend\tagstructend\par

\begingroup\fontsize{10}{12}\selectfont
\begingroup\fontsize{10}{12}\selectfont

\begin{longtable}[t]{r>{\centering\arraybackslash}p{1.57cm}>{\centering\arraybackslash}p{1.57cm}>{\centering\arraybackslash}p{1.57cm}>{\centering\arraybackslash}p{1.57cm}>{\centering\arraybackslash}p{1.57cm}>{\centering\arraybackslash}p{1.57cm}}
\caption{\label{tab:ssbES}Estimated recent trend in spawning biomass and the fraction unfished and the 95 percent intervals.}\\
\toprule
Year & Spawning Biomass (mt) & Lower Interval & Upper Interval & Fraction Unfished & Lower Interval & Upper Interval\\
\midrule
\endfirsthead
\caption[]{Estimated recent trend in spawning biomass and the fraction unfished and the 95 percent intervals. \textit{(continued)}}\\
\toprule
Year & Spawning Biomass (mt) & Lower Interval & Upper Interval & Fraction Unfished & Lower Interval & Upper Interval\\
\midrule
\endhead

\endfoot
\bottomrule
\endlastfoot
2009 & 192063 & 147642.0 & 236484.0 & 0.73 & 0.67 & 0.80\\
2010 & 190648 & 145945.3 & 235350.7 & 0.73 & 0.66 & 0.79\\
2011 & 188543 & 143818.8 & 233267.2 & 0.72 & 0.66 & 0.78\\
2012 & 187010 & 142421.4 & 231598.6 & 0.71 & 0.65 & 0.78\\
2013 & 185558 & 141133.3 & 229982.7 & 0.71 & 0.64 & 0.77\\
2014 & 184397 & 140016.8 & 228777.2 & 0.70 & 0.64 & 0.77\\
2015 & 185026 & 140460.5 & 229591.5 & 0.71 & 0.64 & 0.77\\
2016 & 186878 & 141889.0 & 231867.0 & 0.71 & 0.65 & 0.78\\
2017 & 189095 & 143515.2 & 234674.8 & 0.72 & 0.66 & 0.79\\
2018 & 191353 & 145118.6 & 237587.4 & 0.73 & 0.67 & 0.80\\
2019 & 194068 & 147218.2 & 240917.8 & 0.74 & 0.68 & 0.81\\
2020 & 196572 & 149228.1 & 243915.9 & 0.75 & 0.68 & 0.82\\
2021 & 198983 & 151298.6 & 246667.4 & 0.76 & 0.69 & 0.82\\*
\end{longtable}
\endgroup{}
\endgroup{}


\tagstructbegin{tag=Figure,alttext={Estimated time series of spawning output (circles and line: median; light broken lines: 95 percent intervals) for the base model.}}\tagmcbegin{tag=Figure}

\begin{figure}
\centering
\includegraphics[width=1\textwidth,height=1\textheight]{C:/Assessments/2021/dover_sole_2021/models/7.0.1_base/plots/ts7_Spawning_biomass_(mt)_with_95_asymptotic_intervals_intervals.png}
\caption{Estimated time series of spawning output (circles and line: median; light broken lines: 95 percent intervals) for the base model.\label{fig:es-ssb}}
\end{figure}

\tagmcend\tagstructend

\tagstructbegin{tag=Figure,alttext={Estimated time series of fraction of unfished spawning output (circles and line: median; light broken lines: 95 percent intervals) for the base model.}}\tagmcbegin{tag=Figure}

\begin{figure}
\centering
\includegraphics[width=1\textwidth,height=1\textheight]{C:/Assessments/2021/dover_sole_2021/models/7.0.1_base/plots/ts9_Fraction_of_unfished_with_95_asymptotic_intervals_intervals.png}
\caption{Estimated time series of fraction of unfished spawning output (circles and line: median; light broken lines: 95 percent intervals) for the base model.\label{fig:es-depl}}
\end{figure}

\tagmcend\tagstructend

\clearpage

\tagstructbegin{tag=H2}\tagmcbegin{tag=H2}

\hypertarget{recruitment}{%
\subsection*{Recruitment}\label{recruitment}}
\addcontentsline{toc}{subsection}{Recruitment}

\leavevmode\tagmcend\tagstructend

\tagstructbegin{tag=P}\tagmcbegin{tag=P}

There is large uncertainty associated with annual differences in recruitment across much of the time series due to a lack of informative data during the early period and little contrast in composition and index data in the later period to signal much variation in cohort strength (Table \ref{tab:recrES} and Figure \ref{fig:es-recruits}). Data were most informative from the early-2000s to the mid-2010s, where estimates showed periods of below average recruitment (2002-2006) and above average recruitment (2008-2010). The 2000 and 2009 year classes are estimated to be the largest across the time series and were well determined as being above average (i.e., \textasciitilde95 percent asymptotic intervals did not span 0, Figure \ref{fig:es-rec-devs}). Overall, the Dover sole stock has not been reduced to levels that would provide considerable information on how recruitment changes with across spawning biomass levels (i.e., inform the steepness parameter). Thus, all recruitment is based on a fixed assumption about steepness ({\tagstructbegin{tag=Formula}\tagmcbegin{tag=Formula}\(h\)\leavevmode\tagmcend\tagstructend} = 0.80) and recruitment variability ({\tagstructbegin{tag=Formula}\tagmcbegin{tag=Formula}\(\sigma_R\)\leavevmode\tagmcend\tagstructend} = 0.35).

\leavevmode\tagmcend\tagstructend\par

\begingroup\fontsize{10}{12}\selectfont
\begingroup\fontsize{10}{12}\selectfont

\begin{longtable}[t]{r>{\centering\arraybackslash}p{1.57cm}>{\centering\arraybackslash}p{1.57cm}>{\centering\arraybackslash}p{1.57cm}>{\centering\arraybackslash}p{1.57cm}>{\centering\arraybackslash}p{1.57cm}>{\centering\arraybackslash}p{1.57cm}}
\caption{\label{tab:recrES}Estimated recent trend in recruitment and recruitment deviations (recruit devs.) and the 95 percent intervals.}\\
\toprule
Year & Recruitment & Lower Interval & Upper Interval & Recruitment Deviations & Lower Interval & Upper Interval\\
\midrule
\endfirsthead
\caption[]{Estimated recent trend in recruitment and recruitment deviations (recruit devs.) and the 95 percent intervals. \textit{(continued)}}\\
\toprule
Year & Recruitment & Lower Interval & Upper Interval & Recruitment Deviations & Lower Interval & Upper Interval\\
\midrule
\endhead

\endfoot
\bottomrule
\endlastfoot
2009 & 334902 & 203336.87 & 466467.1 & 0.52 & 0.24 & 0.80\\
2010 & 256930 & 143345.58 & 370514.4 & 0.25 & -0.09 & 0.60\\
2011 & 204214 & 106755.57 & 301672.4 & 0.02 & -0.37 & 0.41\\
2012 & 238648 & 133045.73 & 344250.3 & 0.18 & -0.16 & 0.53\\
2013 & 161941 & 80743.61 & 243138.4 & -0.21 & -0.64 & 0.21\\
2014 & 166317 & 80897.65 & 251736.4 & -0.19 & -0.63 & 0.24\\
2015 & 199178 & 87485.34 & 310870.7 & -0.02 & -0.51 & 0.47\\
2016 & 205309 & 69562.68 & 341055.3 & 0.00 & -0.61 & 0.61\\
2017 & 206028 & 59156.14 & 352899.9 & -0.01 & -0.67 & 0.66\\
2018 & 208863 & 57978.50 & 359747.5 & 0.00 & -0.68 & 0.67\\
2019 & 209235 & 56008.15 & 362461.8 & 0.00 & -0.69 & 0.69\\
2020 & 209423 & 56073.85 & 362772.2 & 0.00 & -0.69 & 0.69\\
2021 & 209596 & 56137.68 & 363054.3 & 0.00 & -0.69 & 0.69\\*
\end{longtable}
\endgroup{}
\endgroup{}


\tagstructbegin{tag=Figure,alttext={Estimated time series of age-0 recruits (1000s) for the base model with 95 percent intervals.}}\tagmcbegin{tag=Figure}

\begin{figure}
\centering
\includegraphics[width=1\textwidth,height=1\textheight]{C:/Assessments/2021/dover_sole_2021/models/7.0.1_base/plots/ts11_Age-0_recruits_(1000s)_with_95_asymptotic_intervals.png}
\caption{Estimated time series of age-0 recruits (1000s) for the base model with 95 percent intervals.\label{fig:es-recruits}}
\end{figure}

\tagmcend\tagstructend

\tagstructbegin{tag=Figure,alttext={Estimated time series of recruitment deviations.}}\tagmcbegin{tag=Figure}

\begin{figure}
\centering
\includegraphics[width=1\textwidth,height=1\textheight]{C:/Assessments/2021/dover_sole_2021/models/7.0.1_base/plots/recdevs2_withbars.png}
\caption{Estimated time series of recruitment deviations.\label{fig:es-rec-devs}}
\end{figure}

\tagmcend\tagstructend

\clearpage

\tagstructbegin{tag=H2}\tagmcbegin{tag=H2}

\hypertarget{exploitation-status}{%
\subsection*{Exploitation Status}\label{exploitation-status}}
\addcontentsline{toc}{subsection}{Exploitation Status}

\leavevmode\tagmcend\tagstructend

\tagstructbegin{tag=P}\tagmcbegin{tag=P}

Trends in fishing intensity (1 - SPR) largely mirrored that of landings given the relative lack of large variations in annual recruitment such that there was a steady increase from the 1940s to the mid to late 1980s before decreasing to current levels of 0.11 for 2020 (Figure \ref{fig:es-1-spr}). The maximum fishing intensity was 0.45 in 1991, well below the target harvest rate of 0.70 (1 - {\tagstructbegin{tag=Formula}\tagmcbegin{tag=Formula}\(\text{SPR}_{30\%}\)\leavevmode\tagmcend\tagstructend}). Fishing intensity over the past decade has ranged between 0.11 and 0.2 and the exploitation rate has been low (0.01 - 0.02, Table \ref{tab:exploitES}). Current estimates indicate that Dover sole spawning biomass is greater than 3 times higher than the target biomass level ({\tagstructbegin{tag=Formula}\tagmcbegin{tag=Formula}\(\text{SB}_{25\%}\)\leavevmode\tagmcend\tagstructend}), and fishing intensity remains well below the target harvest rate.

\leavevmode\tagmcend\tagstructend\par

\begingroup\fontsize{10}{12}\selectfont
\begingroup\fontsize{10}{12}\selectfont

\begin{longtable}[t]{r>{\centering\arraybackslash}p{1.57cm}>{\centering\arraybackslash}p{1.57cm}>{\centering\arraybackslash}p{1.57cm}>{\centering\arraybackslash}p{1.57cm}>{\centering\arraybackslash}p{1.57cm}>{\centering\arraybackslash}p{1.57cm}}
\caption{\label{tab:exploitES}Estimated recent trend in the spawning potential ratio (SPR), 1-SPR , the exploitation rate, along with the 95 percent intervals.}\\
\toprule
Year & 1-SPR & Lower Interval & Upper Interval & Exploitation Rate & Lower Interval & Upper Interval\\
\midrule
\endfirsthead
\caption[]{Estimated recent trend in the spawning potential ratio (SPR), 1-SPR , the exploitation rate, along with the 95 percent intervals. \textit{(continued)}}\\
\toprule
Year & 1-SPR & Lower Interval & Upper Interval & Exploitation Rate & Lower Interval & Upper Interval\\
\midrule
\endhead

\endfoot
\bottomrule
\endlastfoot
2009 & 29.14 & 24.05 & 34.23 & 0.03 & 0.02 & 0.04\\
2010 & 26.85 & 21.97 & 31.72 & 0.03 & 0.02 & 0.03\\
2011 & 20.91 & 16.83 & 24.99 & 0.02 & 0.02 & 0.02\\
2012 & 20.26 & 16.27 & 24.26 & 0.02 & 0.01 & 0.02\\
2013 & 22.11 & 17.81 & 26.42 & 0.02 & 0.02 & 0.03\\
2014 & 18.71 & 14.92 & 22.49 & 0.02 & 0.01 & 0.02\\
2015 & 18.16 & 14.46 & 21.86 & 0.02 & 0.01 & 0.02\\
2016 & 20.24 & 16.20 & 24.27 & 0.02 & 0.01 & 0.02\\
2017 & 21.09 & 16.91 & 25.27 & 0.02 & 0.01 & 0.02\\
2018 & 17.35 & 13.77 & 20.92 & 0.02 & 0.01 & 0.02\\
2019 & 15.49 & 12.25 & 18.74 & 0.01 & 0.01 & 0.02\\
2020 & 12.70 & 9.98 & 15.42 & 0.01 & 0.01 & 0.01\\*
\end{longtable}
\endgroup{}
\endgroup{}


\tagstructbegin{tag=Figure,alttext={Estimated 1 - relative spawning ratio (SPR) by year for the base model. The management target is plotted as a red horizontal line and values above this reflect harvest in excess of the proxy harvest rate.}}\tagmcbegin{tag=Figure}

\begin{figure}
\centering
\includegraphics[width=1\textwidth,height=1\textheight]{C:/Assessments/2021/dover_sole_2021/models/7.0.1_base/plots/SPR2_minusSPRseries.png}
\caption{Estimated 1 - relative spawning ratio (SPR) by year for the base model. The management target is plotted as a red horizontal line and values above this reflect harvest in excess of the proxy harvest rate.\label{fig:es-1-spr}}
\end{figure}

\tagmcend\tagstructend

\clearpage

\tagstructbegin{tag=H2}\tagmcbegin{tag=H2}

\hypertarget{ecosystem-considerations}{%
\subsection*{Ecosystem Considerations}\label{ecosystem-considerations}}
\addcontentsline{toc}{subsection}{Ecosystem Considerations}

\leavevmode\tagmcend\tagstructend

\tagstructbegin{tag=P}\tagmcbegin{tag=P}

Ecosystem factors have not been explicitly modeled in this assessment but there are several aspects of the California current ecosystem that may impact Dover sole population dynamics and warrant further research. Survival of Dover sole eggs and pelagic larvae that have a protracted pelagic phase are linked to water circulation patterns {\tagstructbegin{tag=Reference}\tagmcbegin{tag=Reference}(King et al. 2011)\leavevmode\tagmcend\tagstructend}. The timing of settlement occurs typically between January and March and is correlated with Ekman transport, positive vertical velocity, and relatively warm bottom temperatures {\tagstructbegin{tag=Reference}\tagmcbegin{tag=Reference}(Toole, Markle, and Donohoe 1997)\leavevmode\tagmcend\tagstructend}. Markle et al.~{\tagstructbegin{tag=Reference}\tagmcbegin{tag=Reference}(1992)\leavevmode\tagmcend\tagstructend} hypothesized that juvenile Dover sole move inshore to nursery habitat by making vertical ascents during the night off bottom until they encounter suitable habitat. Tolimieri et al.~{\tagstructbegin{tag=Reference}\tagmcbegin{tag=Reference}(2020)\leavevmode\tagmcend\tagstructend} identified multiple areas off the coast of southern California that had high densities of young Dover sole. This is consistent with the finding of Toole et al.~{\tagstructbegin{tag=Reference}\tagmcbegin{tag=Reference}(2011)\leavevmode\tagmcend\tagstructend} that juvenile Dover sole 10 - 22 cm tended to move inshore during summer months. As Dover sole grow they generally move offshore into deep waters. Changing water temperature due to climate change may alter the winter onshore Ekman transport which could have impacts on juvenile survival and result in distributional shifts of favorable spawning grounds, or nursery habitats of Dover sole.

\leavevmode\tagmcend\tagstructend\par

\tagstructbegin{tag=H2}\tagmcbegin{tag=H2}

\hypertarget{reference-points}\)\leavevmode\tagmcend\tagstructend}) across all model years are shown in Figure \ref{fig:es-phase} where warmer colors (red) represent early years and colder colors (blue) represent recent years. The relative biomass and estimated SPR have been well above the management biomass target (25 percent) and well below the SPR target across all model years. Figure \ref{fig:es-yield} shows the equilibrium curve based on a steepness value fixed at 0.8 with vertical dashed lines to indicate the estimate of fraction unfished at the start of 2021 (current) and the estimated management targets calculated based on the relative target biomass (B target), the SPR target, and the maximum sustainable yield (MSY).

\leavevmode\tagmcend\tagstructend\par

\tagstructbegin{tag=Figure,alttext={Phase plot of estimated 1-SPR versus fraction unfished for the base model.}}\tagmcbegin{tag=Figure}

\begin{figure}
\centering
\includegraphics[width=1\textwidth,height=1\textheight]{C:/Assessments/2021/dover_sole_2021/models/7.0.1_base/plots/SPR4_phase.png}
\caption{Phase plot of estimated 1-SPR versus fraction unfished for the base model.\label{fig:es-phase}}
\end{figure}

\tagmcend\tagstructend

\tagstructbegin{tag=Figure,alttext={Equilibrium yield curve for the base case model. Values are based on the 2020 fishery selectivities and with steepness fixed at 0.80.}}\tagmcbegin{tag=Figure}

\begin{figure}
\centering
\includegraphics[width=1\textwidth,height=1\textheight]{C:/Assessments/2021/dover_sole_2021/models/7.0.1_base/plots/yield2_yield_curve_with_refpoints.png}
\caption{Equilibrium yield curve for the base case model. Values are based on the 2020 fishery selectivities and with steepness fixed at 0.80.\label{fig:es-yield}}
\end{figure}

\tagmcend\tagstructend

\tagstructbegin{tag=P}\tagmcbegin{tag=P}

Reference points were calculated using the estimated selectivities and catch distributions among fleets in the most recent year of the model, 2020 (Table \ref{tab:es-reference}). Sustainable total yield, landings plus discards, using an {\tagstructbegin{tag=Formula}\tagmcbegin{tag=Formula}\(\text{SPR}_{30\%}\)\leavevmode\tagmcend\tagstructend} is 22,891 mt. The spawning biomass equivalent to 25 percent of the unfished spawning biomass ({\tagstructbegin{tag=Formula}\tagmcbegin{tag=Formula}\(\text{SB}_{25\%}\)\leavevmode\tagmcend\tagstructend}) calculated using the SPR target ({\tagstructbegin{tag=Formula}\tagmcbegin{tag=Formula}\(\text{SPR}_{30\%}\)\leavevmode\tagmcend\tagstructend}) was 74,498 mt. Recent removals have been below the point estimate of the potential long-term yields calculated using an {\tagstructbegin{tag=Formula}\tagmcbegin{tag=Formula}\(\text{SPR}_{30\%}\)\leavevmode\tagmcend\tagstructend} reference point and the population scale has been relatively stable or increasing over the last decade.

\leavevmode\tagmcend\tagstructend\par

\begingroup\fontsize{10}{12}\selectfont
\begingroup\fontsize{10}{12}\selectfont

\begin{longtable}[t]{r>{\centering\arraybackslash}p{2cm}>{\centering\arraybackslash}p{2cm}>{\centering\arraybackslash}p{2cm}}
\caption{\label{tab:referenceES}Summary of reference points and management quantities, including estimates of the  95 percent intervals.}\\
\toprule
 & Estimate & Lower Interval & Upper Interval\\
\midrule
\endfirsthead
\caption[]{Summary of reference points and management quantities, including estimates of the  95 percent intervals. \textit{(continued)}}\\
\toprule
 & Estimate & Lower Interval & Upper Interval\\
\midrule
\endhead

\endfoot
\bottomrule
\endlastfoot
Unfished Spawning Biomass (mt) & 230042.00 & 205449.55 & 254634.45\\
Unfished Age 3+ Biomass (mt) & 498521.00 & 445742.50 & 551299.50\\
Unfished Recruitment (R0) & 170843.00 & 152579.10 & 189106.90\\
Spawning Biomass (mt) (2021) & 154873.00 & 122440.30 & 187305.70\\
Fraction Unfished (2021) & 0.67 & 0.60 & 0.75\\
Reference Points Based SB25 Percent & NA & NA & NA\\
Proxy Spawning Biomass (mt)SB25 Percent & 57510.60 & 51362.47 & 63658.73\\
SPR Resulting in SB25 Percent & 0.30 & 0.30 & 0.30\\
Exploitation Rate Resulting in SB25 Percent & 0.11 & 0.10 & 0.11\\
Yield with SPR Based On SB25 Percent (mt) & 18029.30 & 16135.49 & 19923.11\\
Reference Points Based on SPR Proxy for MSY & NA & NA & NA\\
Proxy Spawning Biomass (mt) (SPR30) & 58277.40 & 52047.30 & 64507.50\\
SPR30 & 30.00 & NA & NA\\
Exploitation Rate Corresponding to SPR30 & 0.11 & 0.10 & 0.11\\
Yield with SPR30 at SB SPR (mt) & 18018.80 & 16126.33 & 19911.27\\
Reference Points Based on Estimated MSY Values & NA & NA & NA\\
Spawning Biomass (mt) at MSY (SB MSY) & 53315.50 & 47673.94 & 58957.06\\
SPR MSY & 0.28 & 0.28 & 0.28\\
Exploitation Rate Corresponding to SPR MSY & 0.11 & 0.11 & 0.11\\
MSY (mt) & 18056.50 & 16158.47 & 19954.53\\*
\end{longtable}
\endgroup{}
\endgroup{}


\clearpage

\tagstructbegin{tag=H2}\tagmcbegin{tag=H2}

\hypertarget{management-performance}{%
\subsection*{Management Performance}\label{management-performance}}
\addcontentsline{toc}{subsection}{Management Performance}

\leavevmode\tagmcend\tagstructend

\tagstructbegin{tag=P}\tagmcbegin{tag=P}

Exploitation on Dover sole slowly increased starting around 1940 and reached a high in the early 1990s. After peaking in the 1990s exploitation rates declined steadily through 2006, increased from 2007 - 2010, but have steadily declined since. In the last ten years the annual catch limit (ACL) has been set well below the overfishing limit (OFL) and acceptable biological catch (ABC) (Table \ref{tab:ofl-es}). Total mortality has ranged between 10 - 15 percent of the ACL in the most recent five years.

\leavevmode\tagmcend\tagstructend\par

\begingroup\fontsize{10}{12}\selectfont
\begingroup\fontsize{10}{12}\selectfont

\tagstructbegin{tag=Table}\tagmcbegin{tag=Table}
\begin{longtable}[t]{l>{\raggedright\arraybackslash}p{1.83cm}>{\raggedright\arraybackslash}p{1.83cm}>{\raggedright\arraybackslash}p{1.83cm}>{\raggedright\arraybackslash}p{1.83cm}>{\raggedright\arraybackslash}p{1.83cm}}
\caption{\label{tab:ofl-es}The OFL, ABC, ACL, landings, and the estimated total mortality in metric tons.}\\
\toprule
Year & OFL & ABC & ACL & Landings & Est. Total Mortality\\
\midrule
\endfirsthead
\caption[]{\label{tab:ofl-es}The OFL, ABC, ACL, landings, and the estimated total mortality in metric tons. \textit{(continued)}}\\
\toprule
Year & OFL & ABC & ACL & Landings & Est. Total Mortality\\
\midrule
\endhead

\endfoot
\bottomrule
\endlastfoot
2011 & 44400 & 42436 & 25000 & 7782 & 7893\\
2012 & 44826 & 42843 & 25000 & 7328 & 7430\\
2013 & 92955 & 88865 & 25000 & 7970 & 8078\\
2014 & 77774 & 74352 & 25000 & 6449 & 6543\\
2015 & 66871 & 63929 & 50000 & 6327 & 6354\\
2016 & 59221 & 56615 & 50000 & 7318 & 7350\\
2017 & 89702 & 85755 & 50000 & 7892 & 7925\\
2018 & 90282 & 86310 & 50000 & 6421 & 6447\\
2019 & 91102 & 87094 & 50000 & 5767 & 5790\\
2020 & 92048 & 87998 & 50000 & 4688 & 4707\\*
\end{longtable}
\leavevmode\tagmcend\tagstructend\par
\endgroup{}
\endgroup{}

\tagstructbegin{tag=H2}\tagmcbegin{tag=H2}

\hypertarget{unresolved-problems-and-major-uncertainties}{%
\subsection*{Unresolved Problems and Major Uncertainties}\label{unresolved-problems-and-major-uncertainties}}
\addcontentsline{toc}{subsection}{Unresolved Problems and Major Uncertainties}

\leavevmode\tagmcend\tagstructend

\tagstructbegin{tag=P}\tagmcbegin{tag=P}

The base case model was developed with the goal of balancing parsimony with realism and fitting the data. To achieve parsimony, some simplification of model structure was assumed which may impact the interpretation and fit to specific data sets. The maturity-at-length or -at-age analysis conducted for this assessment identified possible differences in Dover sole south and north of Point Reyes. Currently, there is limited information on the movement of Dover sole by latitude or depth which could provide insights into the mechanisms behind these observed differences. Spatial estimates of biomass north and south of Point Reyes, using \gls{s-wcgbt} data averaged across the most recent five years, indicated that approximately 67 percent of the West Coast Dover sole biomass is estimated to be north of Point Reyes. Additionally, in recent years the majority of fishery data have been collected from ports north of Point Reyes, which limits the ability to support additional model complexity. Given the lack of information to inform the structure and parameterization of a spatial model, the base model assumed a single homogeneous population structure at this time. Future research into the biology and movement of Dover sole could facilitate future spatial modeling efforts if found to be the appropriate approach.

\leavevmode\tagmcend\tagstructend\par

\tagstructbegin{tag=P}\tagmcbegin{tag=P}

Uncertainty in natural mortality translates into uncertain estimates of both status and sustainable fishing levels for Dover sole. In the base model, a balance between fixing and estimating this key parameter was achieved by fixing female natural mortality at the median of the prior while estimating the relative difference in male natural mortality. The difference between male and female natural mortality appeared to be well informed (likelihood profile) with estimates consistent with the data and biology of Dover sole across its range (U.S. west coast, Canada, U.S. Alaska waters). The likelihood profile across values of female natural mortality supported lower values, which were not expected \emph{a priori} based on the available age data and were largely driven by length data from the AFSC slope survey. This could be due to limited information about maximum age for Dover sole in the data, the limited selection of female Dover sole by the fisheries and surveys or could indicate model misspecification. It is unclear what is driving this behavior in the model.

\leavevmode\tagmcend\tagstructend\par

\tagstructbegin{tag=P}\tagmcbegin{tag=P}

Dover sole life history exhibit strong relationships with depth that indicate the stock is more complex than the model assumes. Small fish are found in shallow water, with the median observed size increasing with depth. However, the variability of sizes observed by sex increases moving from deeper to shallower waters. Specifically, the \gls{s-wcgbt} observes large females at the deepest depths sampled but also observe some of the largest female Dover sole in waters less than 300 meters. In addition, there is a pattern of sex ratio by depth with more males being found in middle depths and more females found in shallow and deeper depths. These patterns are apparent in the summer fisheries and surveys. It is uncertain how the patterns affect the data (they may be a cause of the bi-modal length distributions seen in the slope surveys) and if these patterns can be effectively modeled to produce better fits to the data and better predictions of biomass while still preserving model parsimony.

\leavevmode\tagmcend\tagstructend\par

\tagstructbegin{tag=H2}\tagmcbegin{tag=H2}

\hypertarget{scientific-uncertainty}{%
\subsection*{Scientific Uncertainty}\label{scientific-uncertainty}}
\addcontentsline{toc}{subsection}{Scientific Uncertainty}

\leavevmode\tagmcend\tagstructend

\tagstructbegin{tag=P}\tagmcbegin{tag=P}

The model estimated uncertainty around the 2021 spawning biomass was {\tagstructbegin{tag=Formula}\tagmcbegin{tag=Formula}\(\sigma\)\leavevmode\tagmcend\tagstructend} = 0.17 and the uncertainty around the OFL was {\tagstructbegin{tag=Formula}\tagmcbegin{tag=Formula}\(\sigma\)\leavevmode\tagmcend\tagstructend} = 0.16. This is likely an underestimate of overall uncertainty because of the necessity to fix several population dynamic parameters (e.g., steepness, recruitment variance, female natural mortality) and no explicit incorporation of model structural uncertainty (although see the decision table for alternative states of nature).

\leavevmode\tagmcend\tagstructend\par

\tagstructbegin{tag=H2}\tagmcbegin{tag=H2}

\hypertarget{harvest-projections-and-decision-table}{%
\subsection*{Harvest Projections and Decision Table}\label{harvest-projections-and-decision-table}}
\addcontentsline{toc}{subsection}{Harvest Projections and Decision Table}

\leavevmode\tagmcend\tagstructend

\tagstructbegin{tag=P}\tagmcbegin{tag=P}

The forecast of stock abundance and yield was developed using the base model. The total catches in 2021 and 2022 were set at 10,000 mt, well below the adopted 50,000 mt ACL for those year, based on recommendations from the Groundfish Management Team (GMT). These assumed removals are likely higher than what the true removals may be in 2021 and 2023 but have limited impact in the stock status and future removals during the projected period in the base model. The exploitation rate for 2023 and beyond is based upon an SPR of 30 percent and the 25:5 harvest control rule. The average exploitation rates, across recent years, by fleet were used to distribute catches during the forecast period. The ABC values were estimated using a category 1 time-varying {\tagstructbegin{tag=Formula}\tagmcbegin{tag=Formula}\(\sigma_y\)\leavevmode\tagmcend\tagstructend} starting at 0.50 combined with a P* value of 0.45. The catches in the base model during the projection period, 2023 - 2032 were set equal to the year-specific ABC using the current flatfish harvest control rule, 25:5 (Table \ref{tab:es-project}).

\leavevmode\tagmcend\tagstructend\par

\tagstructbegin{tag=P}\tagmcbegin{tag=P}

The axes of uncertainty in the decision table are based on the uncertainty around female natural mortality. The default category 1 {\tagstructbegin{tag=Formula}\tagmcbegin{tag=Formula}\(\sigma\)\leavevmode\tagmcend\tagstructend} value of 0.50 was used to identify the low and high states of nature relative to the estimated 2021 spawning biomass (i.e., 1.15 standard deviations corresponding to the 12.5 and 87.5 percentiles). A search across female natural mortality values was done to identify the natural mortality value that resulted in current year spawning biomass values for the low and high states of nature based on the percentiles. The female natural mortality values that corresponded with the lower and upper percentiles were 0.084 yr\textsuperscript{-1} and 0.126 yr\textsuperscript{-1}.

\leavevmode\tagmcend\tagstructend\par

\tagstructbegin{tag=P}\tagmcbegin{tag=P}

Initial explorations were conducted using the model estimated uncertainty around 2021 spawning biomass of {\tagstructbegin{tag=Formula}\tagmcbegin{tag=Formula}\(\sigma\)\leavevmode\tagmcend\tagstructend} = 0.17 rather than the higher default category 1 {\tagstructbegin{tag=Formula}\tagmcbegin{tag=Formula}\(\sigma\)\leavevmode\tagmcend\tagstructend} value. However, the range of the low and high states of nature relative to the base model were determined to not adequately capture uncertainty based on feedback received during the STAR panel review. Model estimated uncertainty is an underestimate of the true uncertainty around the stock size since it only captures within model uncertainty and does not account for structural uncertainties. Applying a higher {\tagstructbegin{tag=Formula}\tagmcbegin{tag=Formula}\(\sigma\)\leavevmode\tagmcend\tagstructend} value allowed the low and high states of nature to capture a larger uncertainty range around the base model which may be more in line with the cumulative model and structural uncertainty. It was noted that the low and high states of nature results in catchability values (low state of nature catchability = 2.0 and high state of nature catchability = 0.56) for the \gls{s-wcgbt} that were factors higher or lower than the base model catchability (1.072). Catchability values could potentially provide understanding of the plausibility of alternative states; however, adequately interpreting values of catchability comes with inherent challenges due to changes in other key model parameters (e.g., selectivity).

\leavevmode\tagmcend\tagstructend\par

\tagstructbegin{tag=P}\tagmcbegin{tag=P}

Three alternative catch streams were created for the decision table (Table \ref{tab:es-dec-tab}). The first option uses ABC values which are adjusted based on time-varying {\tagstructbegin{tag=Formula}\tagmcbegin{tag=Formula}\(\sigma_y\)\leavevmode\tagmcend\tagstructend} starting at 0.50 and increasing annually combined with a P{\tagstructbegin{tag=Formula}\tagmcbegin{tag=Formula}\(^*\)\leavevmode\tagmcend\tagstructend} value of 0.45. The two alternative catch streams assume fixed catches of either 7,000 or 20,000 mt for the 10 year projection period. All of these options assume full attainment of the catch values.

\leavevmode\tagmcend\tagstructend\par

\tagstructbegin{tag=P}\tagmcbegin{tag=P}

Across the low and high states of nature and across alternative future harvest scenarios the fraction of unfished ranges between 0.023 - 0.895 by the end of the 10 year projection period (Table \ref{tab:es-dec-tab}). The low state of nature assuming full ABC removals results in a nearly depleted stock at the end of the time series. This is due to the assumption or removing the full ABC derived from the base model to the low state of nature which had an overall lower unfished spawning biomass associated with a low natural mortality value which results in a more depleted stock in 2021 relative to the base model.

\leavevmode\tagmcend\tagstructend\par

\begingroup\fontsize{10}{12}\selectfont

\begin{landscape}\begingroup\fontsize{10}{12}\selectfont

\tagstructbegin{tag=Table}\tagmcbegin{tag=Table}
\begin{longtable}[t]{>{\raggedright\arraybackslash}p{2cm}>{\raggedright\arraybackslash}p{2cm}>{\raggedright\arraybackslash}p{2cm}>{\raggedright\arraybackslash}p{2cm}>{\raggedright\arraybackslash}p{2cm}>{\raggedright\arraybackslash}p{2cm}>{\raggedright\arraybackslash}p{2cm}>{\raggedright\arraybackslash}p{2cm}>{\raggedright\arraybackslash}p{2cm}>{\raggedright\arraybackslash}p{2cm}}
\caption{\label{tab:es-project}Projections of potential OFLs (mt), ABCs (mt), the buffer (ABC = buffer x OFL), estimated spawning biomass, and fraction unfished. The adopted OFL, ABC, and ACL for 2021 and 2022 reflect adopted management limits and the assumed removal is the removal assumptions applied for 2021 and 2022. The full ABC was assumed to be removed for 2023 - 2032}\\
\toprule
Year & Adopted OFL (mt) & Adopted ABC (mt) & Adopted ACL (mt) & Assumed Removal (mt) & OFL (mt) & ABC (mt) & Buffer & Spawning Biomass (mt) & Fraction Unfished\\
\midrule
\endfirsthead
\caption[]{\label{tab:es-project}Projections of potential OFLs (mt), ABCs (mt), the buffer (ABC = buffer x OFL), estimated spawning biomass, and fraction unfished. The adopted OFL, ABC, and ACL for 2021 and 2022 reflect adopted management limits and the assumed removal is the removal assumptions applied for 2021 and 2022. The full ABC was assumed to be removed for 2023 - 2032 \textit{(continued)}}\\
\toprule
Year & Adopted OFL (mt) & Adopted ABC (mt) & Adopted ACL (mt) & Assumed Removal (mt) & OFL (mt) & ABC (mt) & Buffer & Spawning Biomass (mt) & Fraction Unfished\\
\midrule
\endhead

\endfoot
\bottomrule
\endlastfoot
2021 & 93547 & 84192 & 50000 & 10000 & - & - & - & 232065 & 0.79\\
2022 & 87540 & 78436 & 50000 & 10000 & - & - & - & 231642 & 0.79\\
2023 & - & - & - & - & 63834 & 59684 & 0.935 & 230918 & 0.79\\
2024 & - & - & - & - & 55859 & 51949 & 0.93 & 207333 & 0.71\\
2025 & - & - & - & - & 49608 & 45937 & 0.926 & 187284 & 0.64\\
2026 & - & - & - & - & 44769 & 41277 & 0.922 & 170449 & 0.58\\
2027 & - & - & - & - & 41053 & 37646 & 0.917 & 156459 & 0.53\\
2028 & - & - & - & - & 38217 & 34892 & 0.913 & 144943 & 0.49\\
2029 & - & - & - & - & 36050 & 32770 & 0.909 & 135500 & 0.46\\
2030 & - & - & - & - & 34389 & 31088 & 0.904 & 127779 & 0.43\\
2031 & - & - & - & - & 33108 & 29797 & 0.9 & 121483 & 0.41\\
2032 & - & - & - & - & 32100 & 28762 & 0.896 & 116323 & 0.40\\*
\end{longtable}
\leavevmode\tagmcend\tagstructend\par
\endgroup{}
\end{landscape}
\endgroup{}

\input{tex_tables/decision_table_es.tex}

\clearpage

\tagstructbegin{tag=H2}\tagmcbegin{tag=H2}

\hypertarget{research-and-data-needs}{%
\subsection*{Research and Data Needs}\label{research-and-data-needs}}
\addcontentsline{toc}{subsection}{Research and Data Needs}

\leavevmode\tagmcend\tagstructend

\tagstructbegin{tag=P}\tagmcbegin{tag=P}

Investigating and or addressing the following items could improve future assessments of Dover sole:

\leavevmode\tagmcend\tagstructend\par

\begin{itemize}

\item Spatiotemporal distribution patterns with depth:  There are patterns of length and sex ratios with depth which may indicate that the stock is more complex than currently modeled.  Further research into the causes of these patterns as well as differences between seasons would help with understanding the stock characteristics such that a more realistic model could be built.  This may also provide further insight into migration and help determine if there are localized populations.

\item Stock boundaries: A common question in stock assessments is whether or not the entire stock is being represented. Dover sole live deeper than the range of the fisheries and surveys.  The assessment model attempts to account for out of area biomass through catchability coefficients and selectivity curves, but that portion of the stock is unknown and can only be conjectured.  Research into abundance in deep areas would be useful to verify that the assessment adequately predicts the entire spawning stock of Dover sole.

\item Unavailable biomass: The distribution of Dover sole covers a wide-depth range off the West Coast. Dover sole are observed by the \gls{s-wcgbt} out to 1,280 m, the maximum depth sampled, where the majority of Dover sole observations at these depths are females. The sex-specific movement of Dover sole across depths results in the model estimating that females` are never fully selected (maximum selectivity well below 1.0 or dome-shaped) by the fisheries or the surveys. This results in an assumption that there is some portion of cryptic biomass that is unavailable for selection by the fisheries or observation by the surveys.  Improved understanding about sex-specific availability across depths by season and the proportion of Dover sole biomass, particularly female biomass, at depths beyond the range of the survey would improve future estimates of stock size. 

\item California Sampling for Ages: Since 1990, nearly 60 percent of fish aged have been landed at the Crescent City port with some years all aged fish being landed there. In contrast, the majority of Dover sole landed in California occur at the Eureka port (approximately 67 percent over the last 10 years). Ensuring that sampling is spread across California ports and otoliths selected for ageing are spread across ports proportional to area removals may provide additional insights to area-specific population attributes.

\end{itemize}

\clearpage

\tagstructbegin{tag=H1}\tagmcbegin{tag=H1}

\hypertarget{references}{%
\section{References}\label{references}}

\leavevmode\tagmcend\tagstructend

\tagstructbegin{tag=BibEntry}\tagmcbegin{tag=BibEntry}

\hypertarget{refs}{}
\begin{cslreferences}
\leavevmode\hypertarget{ref-king_climate_2011}{}%
King, Jacquelynne R., Vera N. Agostini, Christopher J. Harvey, Gordon A. McFarlane, Michael G. G. Foreman, James E. Overland, Emanuele Di Lorenzo, Nicholas A. Bond, and Kerim Y. Aydin. 2011. ``Climate Forcing and the California Current Ecosystem.'' \emph{ICES Journal of Marine Science} 68 (6): 1199--1216. \url{https://doi.org/10.1093/icesjms/fsr009}.

\leavevmode\hypertarget{ref-markle_metamorphosis_1992}{}%
Markle, Douglas F, Phillip M Harris, and Christopher L Toole. 1992. ``Metamorphosis and an Overview of Early-Life-History Stages in Dover Sole Microstomus Pacificus.'' \emph{Fish Bulletin} 90: 285--301.

\leavevmode\hypertarget{ref-tolimieri_spatio-temporal_2020}{}%
Tolimieri, Nick, John Wallace, and Melissa Haltuch. 2020. ``Spatio-Temporal Patterns in Juvenile Habitat for 13 Groundfishes in the California Current Ecosystem.'' Edited by George Tserpes. \emph{PLOS ONE} 15 (8): e0237996. \url{https://doi.org/10.1371/journal.pone.0237996}.

\leavevmode\hypertarget{ref-toole_seasonal_2011}{}%
Toole, C. L., R. D. Brodeur, C. J. Donohoe, and D. F. Markle. 2011. ``Seasonal and Interannual Variability in the Community Structure of Small Demersal Fishes Off the Central Oregon Coast.'' \emph{Marine Ecology Progress Series} 428 (May): 201--17. \url{https://doi.org/10.3354/meps09028}.

\leavevmode\hypertarget{ref-toole_settlement_1997}{}%
Toole, C. L., D. F. Markle, and C. J. Donohoe. 1997. ``Settlement Timing, Distribution, and Abundance of Dover Sole (Microstomus Pacificus) on an Outer Continental Shelf Nursery Area.'' \emph{Canadian Journal of Fisheries and Aquatic Sciences} 54 (3): 531--42. \url{https://doi.org/10.1139/f96-304}.
\end{cslreferences}

\leavevmode\tagmcend\tagstructend
\end{document}
